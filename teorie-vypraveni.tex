\emph{Narratology}, or narrative theory, is a literary science that has experienced considerable development over the last hundred years. The topic of narrative theory has been explored by literary theorists across Europe throughout the twentieth century. Consequently, many different typologies and proposals have emerged on how to view the topic of narrative and narrative mode and how to conduct narrative analysis. \cite{kubicek-vypravec}

In this chapter, I introduce some basic narratological concepts. Next, I show a selected typology to demonstrate its relevance to this thesis and focus on the application of narrative theory to creative writing.

\section{Forms of Narrative by Person}
\label{forms-of-narrative}

\emph{Person} is a linguistic category of verbs, based on which the primary forms of narrative are distinguished. In addition to verbs, this form is also manifested in personal and possessive pronouns and some particular conjunctions. In other words, it is a mode of expression that answers the question of \emph{how} we will write the story, grammatically speaking.\cite{docekalova}

\paragraph{First-person narrative} is a mode of literary narrative in which the narrator tells the story in the first-person singular. Traditionally, first-person narrative reduces the narrator's omniscience to the subjective perspective, often given by the main character. \cite{vlasin-slovnik}

Nevertheless, there are several modes of first-person narration. These modes differ in the level of subjectivity. \cite{dolezel-narativni-zpusoby}

\paragraph{Third-person narrative} is a mode that is realized in the third person singular. Unlike the first-person narrative, the narrator is traditionally in the role of an objective observer. However, in modern literature, we can encounter different levels of objectivity. Therefore, the differences between the semantic aspects of first and third-person narratives are suppressed. Consequently, the main distinction depends on the syntactic structures of these forms. \cite{vlasin-slovnik}

\paragraph{Other modes of narrative} also exist. There is the \emph{second-person narrative}, which is a less frequent narrative mode. We can also encounter narratives given in plural, but these methods are rarely used.

\section{System of Narrative Modes}

From a number of narrative typologies, I decided to choose the system proposed by Czech linguist and literary theorist Lubomír Doležel. Doležel's system of narrative modes is captured in the tree diagram in figure \ref{fig:schema-dolezel}.\cite{dolezel-narativni-zpusoby}

\begin{figure}[ht]
\includegraphics[width=1\textwidth]{data/dolezel-schema.pdf}
\caption{System of narrative modes by Doležel}
\label{fig:schema-dolezel}
\end{figure}

In the section \ref{forms-of-narrative}, I mention some degrees of subjectivity and objectivity of the narrator. The typology offered by Lubomír Doležel is built on the degree of subjectivity and the narrative's person. Similar classifications can be found in various creative writing methodologies \cite{docekalova} since the typology is graspable and intuitive for writers.

In this section, I describe the characteristics of these narrative types, and I demonstrate the differences on a short piece of Czech literary text. \footnote{The text sample is taken from a short story collection \emph{Nejkrásnější dárek (The most beautiful gift) \cite{nejkrasnejsi-darek}} and rewrited for demonstrative purposes. The author of the particular short story is the author of this thesis.}

\subsection{Objective third-person narrative}
This type of narrative mode is written in third-person, and the narrator is standing outside the story. The narrator can be considered an objective observer of the scene who describes what they see. They have no information about the emotions or inner thoughts of the characters. \cite{docekalova} In addition to that, the narrator is free of opinions, and it is not their place to make any judgments.

In the following paragraph, I present a slightly modified part of the short story, which illustrates the characteristics of the objective third-person narrative.
\newline

\begin{otherlanguage*}{czech}
\begin{quoting}
Matyáš pověsil kabát na dřevěný věšák u vchodu, očima vyhledal modré křeslo ve tvaru kapky a sedl si na stoličku naproti němu. Rozhlédl se po kavárně. U okna seděla dívka s chlapcem, drželi se za ruce. V rohu pod policí knih posedávala postarší žena začtená do jednoho z výtisků. A nakonec světlovlasá baristka silnější postavy a tvářemi tak rudými, jako by to byla ona, kdo právě unikl náruči prosincové noci. S nápojovým lístkem v rukou přistoupila k němu, beze slova mu ho podala a zase se vrátila na své místo za pultem, kde se s nezaujatým výrazem věnovala svému mobilu.
\newline
\end{quoting}
\end{otherlanguage*}


As shown above, the narrator is in the role of a reporter. They give testimony about what the characters are doing and how the environment looks. However, the reader gets no mention of the thoughts of neither the narrator nor the characters described.

\subsection{Rhetorical third-person narrative}
The rhetorical third-person narrative shares several common features with the objective one, but it introduces a certain degree of subjectivity into the objective text. Nevertheless, this subjectivity comes from an anonymous narrator, not a particular character. \cite{dolezel-narativni-zpusoby}
\newline

\begin{otherlanguage*}{czech}
\begin{quoting}
Matyáš pověsil kabát na dřevěný věšák u vchodu, očima vyhledal modré křeslo ve tvaru kapky a sedl si na stoličku naproti němu, \textbf{ačkoliv se mu na ní muselo sedět dost nepohodlně}. Rozhlédl se po kavárně. U okna seděla dívka s chlapcem, drželi se za ruce. V rohu pod policí knih posedávala postarší žena, která byla tak začtená do jednoho z výtisků, \textbf{že si nejspíš vůbec nevšimla, že někdo přišel}. A nakonec světlovlasá, \textbf{docela hezká} baristka silnější postavy a tvářemi tak rudými, jako by to byla ona, kdo právě unikl náruči prosincové noci. S nápojovým lístkem v rukou přistoupila k němu, beze slova mu ho podala a zase se vrátila na své místo za pultem, kde se s nezaujatým výrazem věnovala svému mobilu. \textbf{Zdálo se, že se nemůže dočkat, až jí skončí směna}.
\newline
\end{quoting}
\end{otherlanguage*}

As we can see, the narrator makes several judgments in this version and expresses their opinions. However, they still do not know about the characters' inner thoughts, and the subjective expressions are only their assumptions.

\subsection{Subjective third-person narrative}
TODO, viz Doležel, Jan Štětka, Holý (v resources)
zmínit vševědoucnost a nějakou personalitu
- autentifikace
- friedman (Vypravěč, TK)

\subsection{Objective first-person narrative}

The objective type of first-person narrative mode is sporadic. A narrator's desubjectification can appear unnaturally, and the first-person features are only grammatical. The narrator is in the role of an unbiased witness standing outside the story without any judgments. \cite{dolezel-narativni-zpusoby}
\newline

\begin{otherlanguage*}{czech}
\begin{quoting}
Matyáš pověsil kabát na dřevěný věšák u vchodu, očima vyhledal modré křeslo ve tvaru kapky a sedl si na stoličku naproti němu. Rozhlédl se po kavárně. U okna seděla dívka s chlapcem, drželi se za ruce. V rohu pod policí knih posedávala postarší žena začtená do jednoho z výtisků. A nakonec světlovlasá baristka silnější postavy a tvářemi tak rudými, jako by to byla ona, kdo právě unikl náruči prosincové noci. S nápojovým lístkem \textbf{s logem našeho města} v rukou přistoupila k němu, beze slova mu ho podala a zase se vrátila na své místo za pultem, kde se s nezaujatým výrazem věnovala svému mobilu.
\newline
\end{quoting}
\end{otherlanguage*}

The text sample does not differ much from the text in objective third-person since there are few situations where the grammatical person could be manifested. To demonstrate it, I used the words \emph{našeho města (our city)} to expand a sentence. Because of the absence of the narrator's representation, a reader would naturally consider the author as the narrator.

\subsection{Rhetorical first-person narrative}
As in the objective type, the narrator is passive in the story events. However, they expand the events with their views and opinions. Plus, the passivity is not absolute. The narrator touches the story in some way --- they have to receive information about the characters and the environment, get to the place of the story, etc. Nevertheless, they are not part of the narrated story line.\cite{dolezel-narativni-zpusoby}\newline

\begin{otherlanguage*}{czech}
\begin{quoting}
Matyáš pověsil kabát na dřevěný věšák u vchodu, očima vyhledal modré křeslo ve tvaru kapky a sedl si na stoličku naproti němu. Rozhlédl se po kavárně. U okna \textbf{přímo přede mnou} seděla \textbf{docela sympatická} dívka s chlapcem, drželi se za ruce. V rohu pod policí knih posedávala postarší žena -- \textbf{myslím, že jí mohlo být tak sedmdesát} -- začtená do jednoho z výtisků. A nakonec světlovlasá baristka silnější postavy a tvářemi tak rudými, jako by to byla ona, kdo právě unikl náruči prosincové noci. \textbf{Trochu mi připomínala učitelku ze školky.} S nápojovým lístkem v rukou přistoupila k němu, beze slova mu ho podala a zase se vrátila na své místo za pultem, kde se s nezaujatým výrazem věnovala svému mobilu.
\newline
\end{quoting}
\end{otherlanguage*}

In this text sample, we can observe several judgments given by the narrator. It is also expressed that they are sitting in the same room as the characters. However, the short story is not about this narrator, and they figure only as a biased witness.

\subsection{Personal first-person narrative}
Personal type is the most commonly used type of first-person narrative mode. The narrator is an active part of a story, often the main character. They are free to give judgments and express their feelings. The personal narrative thus represents a kind of personal confession of one of the characters. \cite{dolezel-narativni-zpusoby}\newline

\begin{otherlanguage*}{czech}
\begin{quoting}
\textbf{Pověsil jsem} kabát na dřevěný věšák u vchodu, očima vyhledal modré křeslo ve tvaru kapky a sedl si na stoličku naproti němu. \textbf{Rozhlédl jsem} se po kavárně. U okna seděla \textbf{docela sympatická} dívka s chlapcem, drželi se za ruce. V rohu pod policí knih posedávala postarší žena -- \textbf{myslím, že jí mohlo být tak sedmdesát} -- začtená do jednoho z výtisků. A nakonec světlovlasá baristka silnější postavy a tvářemi tak rudými, jako by to byla ona, kdo právě unikl náruči prosincové noci. \textbf{Trochu mi připomínala učitelku ze školky.} S nápojovým lístkem v rukou přistoupila \textbf{ke mně}, beze slova \textbf{mi} ho podala a zase se vrátila na své místo za pultem, kde se s nezaujatým výrazem věnovala svému mobilu. \textbf{Bylo mi jasné, že se nemůže dočkat, až jí skončí směna.} \newline
\end{quoting}
\end{otherlanguage*}

Note that the story is told from Matyáš's point of view in this sample. This fact makes a major difference from rhetorical narratives.

TODO Autentifikace jako u subjective er-form

\section{Point of View}

\section{Impacts of Conversion to Narrative Modes}


TODO omniscience etc. 
