According to scientific estimates, the written word appeared around 3000 BC. The beginnings of the written word gave people the ability to put stories, which until then had only been spread orally, on paper (or its historical equivalent). And this is where the history of creative writing begins. During the millennia, the process of writing evolved. Thanks to the written language, a drafting process was enabled for storytellers.

With the invention of computers, much of the writing process has been automated. Today, we have: the ability to efficiently rewrite a text, functions as Find \& Replace, editors that perform automatic grammar correction, webpages that allow people to collaborate online, word counters, and many other tools that affect the creative process of how the books are being written.

In my eighteen years of writing experience, I have spoken to hundreds of people from the writing community, watching the issues, ideas, and thought processes they -- we -- are dealing with during the drafting process. And there is a question often asked: \emph{how} to write the story? In what person? Who should be the narrator? Sometimes, the writers even rewrite the text to the other person to be able to compare both versions. That made me ask myself another question: could computers also handle this problem?

This thesis aims to automate the process of person change and implement a tool that would convert a written text between first-person and third-person narratives. This conversion covers the grammatical aspects of these narratives. To achieve this goal, I have designed and implemented a rule-based system. That includes an implementation of a tool \emph{RephrasErIch}, proposing several conversion rules for both directions, and then implementing these rules into the tool.

In the second chapter, I briefly summarize the existing solutions for my problem. In the third chapter, I introduce some important terms of the narrative theory and describe the impacts of conversion on a given text's narrative features. \textbf{TODO four, syntactic strucutres}. Chapter five contains an overview of the proposed rules accompanied by diagrams for illustration. In chapter six, I introduce the tool itself. I speak about external tools that I have used for the \emph{RephrasErIch} implementation and then describe the system's design and some important classes.

To assess the tool's performance, I have evaluated it on text data. The evaluation process and its results are described in chapter seven. Afterward, I discuss the errors and the possibility (or impossibility) of solving those errors.

I conclude my thesis with a summarization of achieved results, suggestions for future work, and usage of the implemented tool.

