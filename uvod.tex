According to the scientific estimates we have today, the written word appeared around 3000 BC. The beginnings of the written word gave people the ability to put stories, which until then had only been spread orally, on paper (or its historical equivalent). And this is where the history of creative writing begins. During the millennia, the process of writing evolved. Thanks to the written language, a drafting process was enabled to storytellers.

With the invention of computers, much of the writing process has been automated. Today, we have: the ability to easily rewrite a text, functions as Find \& Replace, editors that perform automatic grammar correction, webpages that allow people to collaborate online, word counters, and many other tools that affect the creative process of how the books are being written.

In my eighteen years of writing experience, I have spoken to hundreds of people from the writing community, watching the issues, ideas and thought processes they -- we -- are dealing with during the drafting process. And there is a question often asked: \emph{how} to write the story? In what person? Who should be the narrator? Sometimes, the writers even rewrite the text to the other person to be able to compare both versions. That made me ask myself ask another question: could computers handle this problem, too?

The goal of this thesis is to automate the process of person change and implement a tool which would convert a written text between first-person and third-person narrative. 




