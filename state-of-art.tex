Problém, kterým se zabývá tato práce, nebyl zatím příliš prozkoumán. V~blasti zpracování přirozeného jazyka (NLP) existuje řada otevřených problémů a přeformulování textui obecně je jedním z~nich. Kromě toho se v~NLP potýkáme se složitou výzvou: mnohá řešení libovolného problémů jsou jazykově specifická. Většina výzkumů, které můžeme najít na jakékoli téma, se přirozeně provádí v~angličtině. Zpracování angličtiny má několik výhod: nejenže ji používají lidé po celém světě, což nám poskytuje obrovské množství jazykových dat, ale navíc je z~lingvistického hlediska díky své jednoduché morfologii snadno zpracovatelná. Naproti tomu čeština je mnohem složitější a vzhledem k~počtu mluvčích se jejím zpracování zabývá významně méně lidí. Jelikož jsem se v~této práci rozhodla pracovat s~českými texty, stojím před touto výzvou i já.

Zmíněné okolnosti nás přivádí k~faktu, že pro češtinu neexistuje žádné řešení tohoto problému, které by mi bylo známo. Najdeme však několik nástrojů, které řeší tuto úlohu alespoň z~části.

Po zobecnění problému na libovolný přirozený jazyk jsem přesto našla jen pár prací na podobné téma, jak se dalo očekávat, všechna pro angličtinu.

V~tét kapitole nejprve představím existující řešení pro angličtinu a poté se zaměřím na situaci v~oblasti zpracování češtiny.

\section{Existující řešení pro angličtinu}

Jak bylo zmíněno v~předchozí kapitole, motivací k~této úloze je vytvořit nástroj pro spisovatele. Nicméně dosavadní práce na toto téma se většinou zaměřují na jiný cíl: převést osobu v~odpovědích poskytovaných virtuálními asistentkami (VA), jako jsou Siri, Alexa nebo Google Assistant. Vzhledem k~odlišnosti cílového využití se tato řešení zabývají některými problémy, které jsou pro tuto práci irelevantní, a naopak opomíjejí několik dílčích úloh souvisejících s~procesem tvůrčího psaní.

Systém, který navrhli Lee et al. \cite{lee2020converting} se zaměřuje na mluvené zprávy v~konverzacích, kde asistuje VA. Pokud je mi známo, jedná se o první pokus konverze úhlu pohledu. V~rámci tohoto výzkumu bylo vyvinuto několik modelů, které můžeme rozdělit podle dvou hlavních přístupů:

\begin{itemize}
	\item[Rule-based přístup] Jde o model, který využívá soubor vytvořených pravidel. Tato pravidla popisují, jak by měl být převod gramaticky proveden.
	\item[Machine learning přístup] Několik dalších modelů je založeno na hlubokém strojovém učení. Tyto modely byly natrénováný na velkém souboru dat pracovníků Amazon Mechanical Turk.
\end{itemize}
Podle výstupů evaluace dosahují lepších výsledků modely založené na strojovém učení.

Nedávno publikovali další snahu o konverzi Granero-Moya a Oikonomu Filandras \cite{granero-moya-oikonomou-filandras-2021-taking}. Jejich výzkum je rovněž zaměřen na technologie virtuálních asistentek. Na rozdíl od mé práce a práce Lee et al. se zaměřují pouze na převod v jednom směru: z~třetí osoby do první osoby. V~publikaci také uvádí oba dva výše zmíněné přístupy. I v~tomto případě vykazují modely založené na hlubokém učení lepší výsledky než rule-based model, který zde posloužil jako baseline.


\section{Nástroje pro češtinu}

V~úvodu této kapitoly bylo objasněno, že pro češtinu v~současné době žádné existující řešení není. Představím tedy stručně několik nástrojů řešících obecnější problémy.

\subsection{Morfologické analyzátory}

Morfologická analýza je nezbytnou součástí zpracování přirozeného jazyka. Je podstatná i pro řešení převodu textu mezi způsoby vyprávění. Například, morfologický analyzátor poskytuje značky, jež popisují gramatické kategorie slova. Tyto značky jsou potřeba k~rozpoznání osoby daného slova.

V~české NLP komunitě existuje několik nástrojů poskytujících morfologickou analýzu..

\paragraph{Ajka}

je jedním z~prvních morfologických analyzátorů pro češtinu. V~současné době se Ajka již nevyvíjí a byla nahrazena novým morfologickým analyzátorem Majka \cite{Sedlacekthesis}

\paragraph{Majka}

je rychlý morfologický analyzátor, který vznikl na Masarykově univerzitě. Implementace navazuje na předchozí systém Ajka, je však rychlejší a flexibilnější. Majka je založena na konečných automatech a pracuje s~morfologickou databází. V~tuto chvíli autor poskytuje databáze nejen pro češtinu, ale i pro další jazyky. Nástroj je schopen přiřadit analyzovanému slovnímu tvaru lemma (základní tvar slova) a značky, vygenerovat všechny slovní tvary daného lemmatu a na základě daného lemmatu a značek vrátit slovní tvar.\cite{majka}

\paragraph{MorphoDiTa} (Morphological Dictionary and Tagger) je nástroj vytvořený na Univerzitě Karlově, který provádí morfologickou analýzu, morfologické generování a značkování. \cite{strakova14}

\subsection{Syntaktické analyzátory}

Kromě morfologie existuje další jazyková rovina, kterou je nutné se při řešení převodu textu mezi osobami zabývat, a tou je syntax. Aby bylo možné větu přeformulovat, je potřeba provést syntaktickou analýzu, navazující na analýzu morfologickou. Stručně představím dva existující syntaktické analyzátory pro češtinu, které byly vyvinuty v~Centru zpracování přirozeného jazyka na Masarykově univerzitě.

\paragraph{SYNT} je syntaktický analyzátor založený na české meta-gramatice a byl navržen pro morfologicky bohaté jazyky..

\paragraph{SET} (Syntactic Engineering Tool), implementuje nový přístup k~syntaktické analýze češtiny, a tím je rozpoznávání vzorů. \cite{set}

