So far, the task of the narrative mode conversion is largely unexplored. There are many open problems in the area of \emph{natural language processing} (NLP), and text rephrasing, in general, is one of them. Besides, there is a specific challenge to face in NLP: several solutions to any NLP task are language-specific. Naturally, most of the research which can be found on any topic is performed on English. Processing of the English language has several advantages: it has far fewer word forms and there is much more of the available textual data in English. These factors make the data more representative, In contrast, Czech morphology is more complex, and due to the number of speakers, there are not many researchers dealing with the NLP  of the Czech language. Since I have chosen to work with Czech texts in this thesis, I have to face this specific challenge as well.

This leads me to the fact that there is no existing solution to this task for Czech. However, there are tools that solve at least some parts of the problem, or similar problems, such as text paraphrasing.

Generalizing the problem to any language, only few papers on this topic are found, as expected, all for English.

In this chapter, at first, I introduce the existing solutions for the English language, and then I examine the situation in the Czech processing. TODO

\section{Existing Solutions for English}

As mentioned in the previous chapter, my motivation for this task is to create a tool for writers. However, previous work on this topic has mostly focused on a different goal: to convert the person in replies provided by \emph{virtual assistants} (VA), such as Siri, Alexa, or Google Assistant. Due to the difference in the target application, these solutions address some problems that are irrelevant to our case and neglect several subtasks related to the creative writing process.

The system designed by Lee et al. \cite{lee2020converting} focused on spoken messages in VA-assisted conversations. To the best of my knowledge, this is the first attempt at \emph{Point of View} (POV) conversion. The authors have developed several models that can be classified by two main approaches: rule-based and machine learning. The rule-based model works with a set of rules which describe how the grammatical conversion should be performed. The model relies on a classification model that provides \emph{Named Entity Recognition} (NER), which is already implemented in the VA. In addition, the model uses a \emph{Part Of Speech} (POS) tagger and a constituency parser.

Besides the rule-based model, several deep learning based models have been developed. These models were trained on data collection provided by Amazon Mechanical Turk workers. According to the evaluation results, the deep learning based models have better performance.

Recently, another effort has been published by Granero Moya and Oikonomu Filandras \cite{granero-moya-oikonomou-filandras-2021-taking}. Their research also focuses on virtual assistant technologies. Unlike my work and the work of Lee et al. they focus on conversion in only one direction: third-person to first-person. They also worked with the two approaches and developed a rule-based model and a deep learning system. Their deep learning models show better results than the rule-based model, which served as a baseline here, as well.

\section{Tools for Czech}

In the introduction to this chapter, I explained that there is currently no solution to this problem for Czech. Nevertheless, I will briefly introduce a few tools that address more general problems.

\subsection{Morphological analyzers}

Morphological analysis is an essential part of NLP. In the context of narrative mode conversion, it cannot be done without morphological analysis. For instance, the person of a word being processed needs to be known, and a morphological analyzer can provide the tags describing the grammatical categories of the word.

There are several tools in the Czech community providing morphological analysis.

\paragraph{Ajka}

is one of the first morphological analyzers for Czech. Currently, Ajka is no longer in development and has been replaced by the new morphological analyzer Majka. \cite{Sedlacekthesis}

\paragraph{Majka}

is a fast morphological analyzer developed at Masaryk University. The implementation builds on the previous system, Ajka, but it is faster and more flexible. Majka is based on finite automata, and it works with a morphological database. At the moment, the author provides databases not only for Czech but also for other languages. The tool is able to assign lemmas and tags to the analyzed word form, generate all word forms of a given lemma and generate a word form based on the given lemma and tags. \cite{majka}

\paragraph{Morfo} is a system developed at Charles University, which provides morphological analysis using a large morphological dictionary. \cite{morfo}


\subsection{Morphological taggers}

Morphological disambiguation, also called tagging, reduces the output of morphological analysis to a single base shape and a single tag that are valid for a token in a particular context.

\paragraph{MorphoDiTa} (Morphological Dictionary and Tagger) is a tool developed at Charles University that performs morphological analysis, morphological generation and tagging. \cite{strakova14}

\paragraph{Desamb} is a morphological tagger developed at Masaryk University, which combines rule-based and statistical methods. \cite{desamb2010}


\subsection{Syntactic analyzers}
\label{sec:synt-an}

In addition to morphology, there is another linguistic level that we need to deal with in order to perform POV conversion: syntax. In order to rephrase a sentence, syntactic analysis is required. The syntactic analysis is preceeded by morphological analysis. Let's take a look at some existing syntactic analyzers for Czech.

\paragraph{SYNT} is a syntactic analyzer developed at the NLP Centre at Masaryk University. The tool is based on Czech meta-grammar, and it was designed for morphologically-rich languages.

\paragraph{SET} (Syntactic Engineering Tool), also developed at the NLP Centre, is implementing a new approach to the syntactic analysis of Czech. This new approach is based on pattern recognition. \cite{set}

\paragraph{UDPipe} is a pipeline from Charles University, which performs tagging, lemmatization and syntactic analysis utilizing neural networks. \cite{straka-2018-udpipe}

\subsection*{Note on the tools listed}

When analysing a text, we usually proceed in order from the lowest levels of language exploration to the higher ones. Therefore, the tools mentioned above are often plugged into pipelines one after the other.

