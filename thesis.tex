\documentclass[digital, oneside, table, nolot, nolof]{fithesis4}
%\documentclass[printed, twoside, table, nolot, nolof]{fithesis4}

%% The following section sets up the locales used in the thesis.
\usepackage[resetfonts]{cmap}
\usepackage[T1]{fontenc}
\usepackage[main=english, czech]{babel}

%% The following section sets up the metadata of the thesis.
\thesissetup{
    date          = \the\year/\the\month/\the\day,
    university    = mu,
    faculty       = fi,
    type          = bc,
    department    = Department of Machine Learning and Data Processing,
    author        = Veronika Burgerová,
    gender        = f,
    advisor       = {RNDr. Zuzana Nevěřilová, Ph.D.},
    title         = {Conversion between First-person and Third-person Narratives},
    TeXtitle      = {Conversion between First-person and Third-person Narratives},
    keywords      = {NLP, text analysis and parsing, text generation, narratology, Czech},
    TeXkeywords   = {NLP, text analysis and parsing, text generation, narratology, Czech},
    abstract      = {Tato práce se zabývá převodem literárního textu mezi ich-formou a er-formou. Cílem je vytvořit nástroj, který převádí texty mezi oběma oběma formami. V práci se zaměřuji na shrnutí základů teorie vyprávění a syntaktických struktur odlišujících ich-formu a er-formu. Na základě těchto informací navrhuji sadu pravidel pro zajištění převodu, popisuji implementaci nástroje provádějícího automatickou konverzi a vyhodnocuji kvalitů dosažených výsledků.

In this thesis, I explore the topic of the conversion between first-person and third-person narratives. The aim is to create a tool that converts texts between the two forms. I summarize the basics of narrative theory and the syntactic structures that distinguish the first and third person. Based on this overview, I propose a set of rules to perform the conversion, describe the implementation of a tool providing automatic conversion, and evaluate the quality of the results achieved.

},
    thanks        = {First of all, I would like to thank my advisor Zuzana Nevěřilová for helping me discover a suitable topic when I thought it was completely hopeless, for her willingness to help me, and many valuable comments on the whole thesis. I also thank all the external evaluators and authors who provided their texts.

Warm thanks to my family for their lifelong support, my friends for keeping me sane, the penguins and Paštika for the daily laugh, and Schola Artist for the opportunity to hang upside down and fly away from everything on aerial silks.

Finally, I cannot put into words my gratitude to Honza Horáček for his undying support, love, care, and unimaginable tolerance. I am grateful to have converted this person into my person.
},
    bib           = bibliography.bib,
    facultyLogo = fithesis-fi,
}
\usepackage{makeidx}      %% The `makeidx` package contains
\makeindex                %% helper commands for index typesetting.
\usepackage[acronym]{glossaries}          %% The `glossaries` package
\renewcommand*\glspostdescription{\hfill} %% contains helper commands
\loadglsentries{example-terms-abbrs.tex}  %% for typesetting glossaries
%\makenoidxglossaries                      %% and lists of abbreviations.
\makeglossaries
%% These additional packages are used within the document:
\usepackage{paralist} %% Compact list environments
\usepackage{amsmath}  %% Mathematics
\usepackage{amsthm}
\usepackage{amsfonts}
\usepackage{url}      %% Hyperlinks
\usepackage{listings} %% Source code highlighting
\lstset{
  basicstyle      = \ttfamily,
  identifierstyle = \color{black},
  keywordstyle    = \color{blue},
  keywordstyle    = {[2]\color{cyan}},
  stringstyle     = \color{teal},
  commentstyle    = \itshape\color{magenta},
  breaklines      = true,
  keywords={,input, output, return, function, in, if, else, foreach, while, begin, end, },
  showstringspaces=false,
}
\usepackage{floatrow} %% Putting captions above tables
\floatsetup[table]{capposition=top}
\usepackage[babel]{csquotes} %% Context-sensitive quotation marks
\usepackage{hyperref}

\usepackage[font=itshape]{quoting}
\begin{document}

% Highlight overfulls
% \setlength{\overfullrule}{5pt} % TODO: remove

\chapter{Introduction} \label{chap:uvod}
According to scientific estimates, the written word appeared around 3000 BC. The beginnings of the written word gave people the ability to put stories, which until then had only been spread orally, on paper (or its historical equivalent). And this is where the history of creative writing begins. During the millennia, the process of writing evolved. Thanks to the written language, a drafting process was enabled for storytellers.

With the invention of computers, much of the writing process has been automated. Today, we have: the ability to efficiently rewrite a text, functions as Find \& Replace, editors that perform automatic grammar correction, webpages that allow people to collaborate online, word counters, and many other tools that affect the creative process of how the books are being written.

In my eighteen years of writing experience, I have spoken to hundreds of people from the writing community, watching the issues, ideas, and thought processes they -- we -- are dealing with during the drafting process. And there is a question often asked: \emph{how} to write the story? In what person? Who should be the narrator? Sometimes, the writers even rewrite the text to the other person to be able to compare both versions. That made me ask myself another question: could computers also handle this problem?

This thesis aims to automate the process of person change and implement a tool that would convert a written text between first-person and third-person narratives. This conversion covers the grammatical aspects of these narratives. To achieve this goal, I have designed and implemented a rule-based system. That includes an implementation of a tool \emph{RephrasErIch}, proposing several conversion rules for both directions, and then implementing these rules into the tool. I am not aiming to create an application usable by ordinary users but to explore the topic of automatic narrative conversion, attempt an experimental implementation of proposed concepts, and evaluate its functionality on text data.

In the second chapter, I briefly summarize the existing solutions for my problem. In the third chapter, I introduce some essential terms of the narrative theory and describe the impacts of conversion on a given text's narrative features. The fourth chapter offers an overview of the syntactic structures of Czech grammar in which narrative conversion is reflected. Chapter five contains an overview of the proposed rules accompanied by diagrams for illustration. In chapter six, I introduce the tool itself. I speak about external tools that I have used for the \emph{RephrasErIch} implementation and then describe the system's design and some important classes.

To assess the tool's performance, I have evaluated it on text data. The evaluation process and its results are described in chapter seven. Afterward, I discuss the errors and the possibility (or impossibility) of solving those errors.

I conclude my thesis with a summarization of achieved results, suggestions for future work, and usage of the implemented tool.



\chapter{State of the Art} \label{chap:state-of-art}
So far, the task of the narrative mode conversion is largely unexplored. There are a lot of open problems in the area of natural language processing (NLP) and text rephrasing in general is one of them. Besides, we have to face a specific challenge in NLP: several solutions to any NLP task are language-specific. Naturally, most of the research we can find on any topic is performed on English. Processing of the English language has several advantages: not only it is widely used and there is a lot of textual data written in English, but from a linguistic perspective it is easy to process due to its simple morphology. In contrast, Czech is much more complex and, due to the number of speakers, there are not many researchers dealing with the natural language processing of the Czech language. Since I have chosen to work with Czech texts in this thesis, I have to face this specific challenge as well.

This leads us to the fact that there is no existing solution to this task for Czech. However, there are tools that solve at least some parts of the problem.

If we generalize the problem to any language, we find only few papers on this topic, as expected, all for English.

In this chapter, at first I introduce the existing solutions for the English language and then I take a look at the situation in the Czech processing.

\section{Existing solutions for English}

As mentioned in the previous chapter, my motivation for this task is to create a tool for writers. However, previous work on this topic has mostly focused on a different goal: to convert the person in replies provided by virtual assistants (VA), such as Siri, Alexa or Google Assistant.  Due to the difference in the target application, these solutions address some problems that are irrelevant to our case, and neglect several subtasks related to the creative writing process.

The system designed by Lee et al. \cite{lee2020converting} focused on spoken messages in VA-assisted conversations. To the best to our knowledge, this is the first attempt at point-of-view conversion. The authors have developed several models that can be classified by two main approaches: rule-based and machine learning. The rule-based model works with a set of rules which describe how the grammatical conversion should be performed. The model relies on classification model that provides Named Entity Recognition, which is already implemented in the VA. In addition, the model is using a Part Of Speech (POS) tagger and a constituency parser. Furthermore, several deep learning based models have been developed. These models were trained on data collection provided by Amazon Mechanical Turk workers. According to the evaluation results, the deep learning based models have better performance.

Recently, another effort for conversion has been published by Granero-Moya and Oikonomu Filandras \cite{granero-moya-oikonomou-filandras-2021-taking}. Their research also focuses on virtual assistant technologies. Unlike our work and the work of Lee at al. they focus on conversion in only one direction: third-person to first-person. They also worked with the two approaches and developed a rule-based model and a deep learning system. Their deep learning models show better results than the rule-based model, that served as a baseline here, as well.

\section{Tools for Czech}

In the introduction to this chapter I explained that there is currently no solution to this problem for Czech. Nevertheless, I will briefly introduce a few tools that address more general problems.

\subsection{Morphological analyzers}

Morphological analysis is an essential part of natural language processing. In the context of narrative mode conversion, we cannot do without morphological analysis. For instance, we need to know the person of a word being processed, and a morphological analyzer can provide us with tags describing the grammatical categories of the word.

There are several tools in the Czech community providing morphological analysis.

\paragraph{Ajka}

is one of the first morphological analyzers for Czech. Currently Ajka is no longer in development and has been replaced by the new morphological analyzer Majka. \cite{Sedlacekthesis}

\paragraph{Majka}

is a fast morphological analyzer that was developed at Masaryk University. The implementation builds on the previous system Ajka, but it is faster and more flexible. Majka is based on finite automata and it works with a morphological database. At the moment, the author provides databases not only for Czech but also for other languages. The tool is able to assign lemmas and tags to the analyzed word form, generate all word forms of a given lemma and generate a word form based on the given lemma and tags. \cite{majka}

\paragraph{MorphoDiTa} (Morphological Dictionary and Tagger) is a tool developed at Charles University that performs morphological analysis, morphological generation and tagging. \cite{strakova14}

\subsection{Syntactic analyzers}

In addition to morphology, there is another linguistic level that we need to deal with in order to perform POV conversion, and that is syntax. In order to rephrase a sentence, syntactic analysis is required. The syntactic analysis is preceded by morphological analysis. Let's take a look at the existing syntactic analyzers for Czech.

\paragraph{SYNT} is a syntactic analyzer developed at the NLP Centre at Masaryk University. The tool is based on Czech meta-grammar and it was designed for morphologically-rich languages.

\paragraph{SET} (Syntactic Engineering Tool), also developed at the NLP Centre, is implementing a new approach to the syntactic analysis of Czech. This new approach is based on pattern recognition. \cite{set}


\chapter{Narrative Theory} \label{chap:teorie-vypraveni}
\emph{Naratologie}, neboli teorie vyprávění, je literární věda, která si během posledních sta let prošla nemalým vývojem. Tématem vyprávěcí teorie se v průběhu dvacátého století zabývali literární teoretici napříč Evropou. Vznikla tedy řada různých typologií a návrhů, jak na téma vypravěče a způsobu vyprávění pohlížet, a jak provádět narativní analýzu.\cite{kubicek-vypravec}

V této kapitole představuji několik základních naratologických pojmů. Dále ukážu vybranou typologii, na níž demonstruji souvislost s touto prací, ????? a zaměřím se na aplikaci vyprávěcí teorie v tvůrčím psaní.

\section{Základy literární teorie a naratologie}

\subsection{Point of view}

Pojem \emph{point of view} lze do češtiny přeložit jako \emph{hledisko}. Hledisko je

TODO najít tu knihu s definicemi pojmů!

\subsection{Formy vyprávění dle osoby}

\emph{Osoba} je mluvnická kategorie sloves v češtině, na základě které se rozlišují základní formy vyprávění. Kromě sloves se tato forma projevuje také v osobních a přivlastňovacích zájmenech.

\paragraph{Ich-forma}

TODO kniha s definicemi

\paragraph{Du-forma}

\paragraph{Er-forma}

\subsection{Systém narativních způsobů}

Z řady narativních typologií jsem se rozhodla vybrat systém, který navrhl jazykovědec a literární teoretik Lubomír Doležel. Doleželův systém vypravěčských způsobů zachycuje stromový graf na obrázku \ref{fig:schema-dolezel}.\cite{dolezel-narativni-zpusoby}

\begin{figure}[ht]
\includegraphics[width=1\textwidth]{data/dolezel-schema.pdf}
\caption{Systém narativních způsobů podle Doležela}
\label{fig:schema-dolezel}
\end{figure}





\chapter{Syntactic Structures in the Narratives} \label{chap:syntax}
Popište syntaktické struktury uplatněné v každé z těchto forem.



\chapter{Conversion Rules} \label{chap:navrh-pravidel}
\section{First-person to Third-person Rules}
In First-person $\rightarrow$ Third-person conversion direction, I have proposed four rules. These four rules cover:
	\begin{itemize}
		\item Personal pronouns replacement
		\item Possesive pronouns replacement
		\item Replacement of conditional, present and future verb forms, conjunctions
		\item Auxiliary verbs replacement or deletion
	\end{itemize}

In this section, I describe each of these rules.

\subsection{Personal pronouns}

This rule covers the conversion of a personal pronoun \emph{já (I)} and its forms. The pronoun can be replaced by:
	\begin{itemize}
		\item another pronoun -- \emph{ona (she)}/\emph{on (he)}
		\item noun -- usually a proper noun, given as the protagonist's name
	\end{itemize}

In both cases, the replacement must be in a corresponding form to keep a sentence grammatically correct.

The rule is illustrated in a diagram \ref{fig:icher-perspron-rule}.

\begin{figure}[!htbp]
\includegraphics[width=\textwidth]{data/Icher-Perspron-Rule.pdf}
\caption{Personal pronouns replacement rule}
\label{fig:icher-perspron-rule}
\end{figure}

\subsection{Possessive pronouns}

In addition to personal pronouns, possessive pronouns must also be converted. The process is similar to the previous one. The goal is to convert a possessive pronoun \emph{můj (my)} and its forms to possessive pronouns \emph{její (her)} / \emph{jeho (his)}, or the possessive form of a proper noun. Considering the limits given by the morphological analyzer, I have decided not to include the second type of replacement. Therefore all the possessive pronouns would be replaced by possessive pronouns.

In Czech, there is also the possessive reflexive \emph{svůj}, which does not depend on the person and always refers to the subject. \cite{Karlik2017} This pronoun is therefore not affected by the conversion.

The rule is illustrated in a diagram \ref{fig:icher-posspron-rule}.

\begin{figure}[!htbp]
\includegraphics[width=\textwidth]{data/Icher-Posspron-Rule.pdf}
\caption{Possessive pronouns replacement rule}
\label{fig:icher-posspron-rule}
\end{figure}

\subsection{Conditionals, indicatives, conjunctions}

The third rule covers several cases because the conversion procedure is the same in all cases. The process is straightforward: it is necessary to change the person in the word tag and then generate a new word form from this new tag and the original lemma, as shown in Figure \ref{fig:icher-predicate-rule}.

This rule includes the following types of conversion:
\begin{itemize}
	\item bych/bychom $\rightarrow$ by -- \emph{conditional auxiliary verbs}
	\item budu/budeme $\rightarrow$ bude/budou -- \emph{future indicatives}
	\item píšu/píšeme $\rightarrow$ píše/píšeme -- \emph{example of present indicative}
	\item abych/abychom/kdybych/kdybychom $\rightarrow$ aby/kdyby -- \emph{conjunctions + conditional auxiliary}
\end{itemize}


\begin{figure}[!htbp]
\includegraphics[width=\textwidth]{data/Icher-Predicate-Rule.pdf}
\caption{Rule replacing the conditionals, conjunctions and verb forms in present indicative tense and future indicative tense}
\label{fig:icher-predicate-rule}
\end{figure}

\subsection{Auxiliary verbs} \label{sec:aux-del}

Finally, it is needed to replace or delete other auxiliary verbs. If the auxiliary verb depends on an active participle, the auxiliar should be deleted. However, if the participle is passive, the auxiliar should be kept and converted.

For example, a sentence \emph{Ukradl \textbf{jsem} klávesnici (I stole a keyboard)} converts to \emph{Ukradl klávesnici (He stole a keyboard)}, but \emph{\textbf{Jsem} ukradena (I am stolen)} should convert to \emph{\textbf{Je} ukradena (She is stolen)}.

Also, the auxiliary verb needs to be in the indicative mode to be converted. The previous rule covers conditional verbs. Then, other modes of auxiliary verbs should not be converted at all, as the auxiliar in plusquamperfect. For instance, a sentence \emph{\textbf{Byl jsem} ukradl klávesnici} might convert to \emph{Juraj \textbf{byl} ukradl klávesnici}. As can be seen, the sentence in the first-person narrative contains two auxiliary verbs; however, only the one in indicative mode would be deleted.

I present the rule in Figure \ref{fig:icher-auxverb-rule}.

\begin{figure}[!htbp]
\includegraphics[width=\textwidth]{data/Icher-Auxverb-Rule.pdf}
\caption{Rule replacing the indicative forms of auxiliary verbs}
\label{fig:icher-auxverb-rule}
\end{figure}

\section{Third-person to First-person Rules}

Since this direction is much more complicated for conversion than the other one, I propose seven rules which cover:

\begin{itemize}
	\item Proper nouns replacement
	\item Predicates replacement
	\item Conditional auxiliars replacement
	\item Auxiliary verbs addition
	\item Personal pronouns replacement
	\item Possessive pronouns replacement
	\item Conjunction replacement
\end{itemize}

\subsection{Proper nouns}

The first rule deals with the occurrences of the protagonist's name. Firstly, it has to decide if the name should be replaced by a personal pronoun or skipped. The name can only be skipped if the member is a subject and the name is not part of multiple subject coordination.

The rule is illustrated in Figure \ref{fig:erich-name-rule}.

\begin{figure}[!htbp]
\includegraphics[width=\textwidth]{data/Erich-Name-Rule.pdf}
\caption{Rule replacing or skipping the occurrences of protagonist's name}
\label{fig:erich-name-rule}
\end{figure}

\subsection{Predicates}

Indicative verbs should be replaced. In contrast to First-person $\rightarrow$ Third-person conversion, the information about a person is not enough. The subject related to the predicate has to be known, and the verb should be replaced by a new form only if the subject refers to the protagonist, as I show in Figure \ref{fig:erich-predicate-rule}

\begin{figure}[!htbp]
\includegraphics[]{data/Erich-Predicate-Rule.pdf}
\caption{Rule covering the predicate replacement}
\label{fig:erich-predicate-rule}
\end{figure}

\subsection{Conditional auxiliars}

These auxiliars are treated similarly to the predicates. I illustrate the decision process in Figure \ref{fig:erich-conditional-rule}


\begin{figure}[!htbp]
\includegraphics[]{data/Erich-Conditional-Rule.pdf}
\caption{Rule replacing the conditional auxiliars}
\label{fig:erich-conditional-rule}
\end{figure}

\subsection{Auxiliars addition}

I talk about auxiliar deletion in Section \ref{sec:aux-del}. Naturally, in the opposite direction, the auxiliars need to be added. Thus, the rule considers the participles. For each predicate expressed as a participle, a subject is found. If the subject refers to the protagonist, an auxiliary verb would be added.

Figure \ref{fig:erich-auxverb-rule} illustrates this rule.


\begin{figure}[!htbp]
\includegraphics[]{data/Erich-Auxverb-Rule.pdf}
\caption{Rule adding the auxiliars to the sentence}
\label{fig:erich-auxverb-rule}
\end{figure}

\subsection{Personal pronouns}

This rule covers the conversion of personal pronouns. The goal is to replace the pronouns which refer to the protagonist with a personal pronoun in the first person, as shown in the Figure \ref{fig:erich-perspron-rule}.

\begin{figure}[!htbp]
\includegraphics[]{data/Erich-Perspron-Rule.pdf}
\caption{Rule replacing the personal pronouns}
\label{fig:erich-perspron-rule}
\end{figure}

\subsection{Possessive pronouns}

As can be seen in Figure \ref{fig:erich-posspron-rule}, the process of replacing possessive pronouns is almost the same as the previous one. Nevertheless, finding whom the pronoun refers to is more complicated.

\begin{figure}[!htbp]
\includegraphics[]{data/Erich-Posspron-Rule.pdf}
\caption{Rule replacing the possessive pronouns}
\label{fig:erich-posspron-rule}
\end{figure}

\subsection{Conjunctions}

The rule is not applied directly to the conjunction but only retrospectively when applying the rule for adding auxiliary verbs, as can be seen in Figure \ref{fig:erich-conjs-rule}.

\begin{figure}[!htbp]
\includegraphics[]{data/Erich-Conjs-Rule.pdf}
\caption{Rule replacing the special conjunctions}
\label{fig:erich-conjs-rule}
\end{figure}


\section{Direct speech}

The last rule is independent of the person, and it covers the processing of direct speech. The simplifying assumption is that direct speech is enclosed by quotes. Therefore, all the text found in quotes should be ignored during the conversion process.

\section*{Conclusion}

In this chapter, I proposed twelve conversion rules covering both directions. The rules mainly cover grammatical phenomena: pronoun and proper noun replacement, changes in verb forms, and modifications of the conjunctions \emph{aby} and \emph{kdyby}.

As explained above, the rule conversion from third person to first person is considerably more complicated. For this reason, not all rules can be implemented, and worse results can be expected in this direction.

In the following chapters, I describe the implementation of the rules into the conversion system and then focus on evaluating the accuracy and coverage of the rules.



\chapter{RephrasErIch} \label{chap:rerich}
%% s využitím syntaktického analyzátoru SET vytvořte program, který bude převádět texty mezi oběma formami. Zohledněte gramaticko-syntaktické jevy spojené s každou z vyprávěcích forem (osoba slovesných tvarů, osobní a přivlastňovací zájmena, spojky kdyby a aby, přímá řeč, rezoluce anafor).
In this chapter, I describe the implementation of the tool \emph{RephrasErIch} which converts a given text between first-person and third-person narratives, and I present other tools that I used in the implementation.

\section{Tools used}

\subsection{SET}

As mentioned before, syntactic analysis is needed for narrative conversion. From the existing Czech syntactic analyzers, I decided to use analyzer \emph{SET}. The main reason why I chose this analyzer is my previous experience with using this tool and its ability to produce dependency trees.

SET is implemented in the Python programming language. Although RephrasErIch is also written in this programming language, it would be challenging to integrate the tools since SET is implemented in the old Python version, Python 2. For this reason, the tools communicate with each other through the command line. The communication is handled by a class \texttt{Syn}.

\subsection{Majka}

\emph{Majka} \cite{majka} provides a morphological analysis. The tool is able to:
\begin{itemize}
	\item assign lemmas and tags to words
	\item generate all correct word forms and tags for a given lemma
	\item generate a word form according to a given lemma and tag
\end{itemize}

RephrasErIch uses Majka directly, mainly for the third use case. Majka is written in C language. Therefore my tool contains a class \texttt{morph} which calls Majka's binary code from a command line and processes the output.

\subsection{Desamb}

\emph{Desamb} is a morphological disambiguator using Majka as a morphological tagger. Majka finds the set of lemmas and tags to the given word, then Desamb chooses the most appropriate lemma and tag. Unlike Majka, results also depend on the word's context, essential for correct tagging.

In my implementation, I use Desamb to generate an input for syntactic analysis.

\subsection{Aara}

The last task that I use external tools for is \emph{anaphora resolution}. Anaphora resolution is a problem of resolving what a pronoun refers to earlier or later in the discourse. For instance, the resolution of a sentence: \emph{Jacob saw his dad in the school} would tell us that \emph{his} refers to \emph{Jacob}.
%% to do: cite

There are few tools for Czech anaphora resolution, and the existing ones do not perform very well. Still, anaphora resolution is required for third-to-first conversion. Thus I decided to use a tool \emph{Aara}. Usage of Aara is provided by a class \texttt{Anaph}, which is run from a command line for the same reasons as SET.

\section{Class structure}

The tool is implemented as a Python library. The library consists of six important classes. In this section, I briefly describe these main classes. Next, I show how they interact in an example input.

\subsection{RephrasErIch}

\texttt{RephrasErIch} is the highest class of the implementation which represents the tool itself. It keeps three attributes:

\begin{itemize}
	\item \texttt{from\_form} -- the narrative form we would like to convert from
	\item \texttt{protg} -- a story protagonist
	\item \texttt{text} -- a given text, an instance of a class \texttt{Text}
\end{itemize}

The essential method is a method \texttt{rephrase} which creates an object representation of the given text and it starts the narrative conversion.

\subsection{Text}

Class \texttt{Text} is an object representation of text data to be converted. When constructed, the text is divided into paragraphs. The paragraphs are being kept as a list of \texttt{Paragraph} objects.

\subsection{Paragraph}

Just as text is divided into paragraphs, a paragraph is divided into sentences. Besides the division, the class \texttt{Paragraph} is responsible for direct speech preprocessing and anaphora resolution.

After the direct speech has been processed and the anaphors resolved, the \texttt{create\_sentences} method is called. This method segments the paragraph and creates a list of \texttt{Sentence} objects.

\subsection{Sentence}

Class \texttt{Sentence} takes a text version of a given sentence and a dictionary of anaphors related to the sentence. At the sentence level, the syntactic analysis is performed. A sentence is represented as a dependency tree generated by \texttt{SET} and \texttt{Desamb}. Every node of this tree (usually a word) is an instance of a class \texttt{Word}.

\subsection{Word}

The basic unit of this recursive class structure is class \texttt{Word}. After performing the higher-level analyses, the word now has all the information needed to be converted: protagonist, anaphors, direct speech, position in the sentence tree, word's lemma, etc.

At the word level, finally, the narrative conversion is being done. The word generates a new form as it applies the earlier proposed rules.

\subsection{Protagonist}

The last important class I have decided to mention is the \texttt{Protagonist}. This class stands out somewhat from the whole class structure, but at the same time, it is passed by across all the other classes. It keeps the information about the story protagonist, which is essential to the conversion process.



\chapter{Evaluation} \label{chap:evaluace}
% Na zvoleném českém korpusu vyhodnoťte přesnost a pokrytí pravidel. Diskutujte míru, se kterou je možné generovat gramatické věty, a kvalitu textu (přirozené vyjadřování, nezamýšlené vedlejší efekty).

\section{Evaluation Corpus}

For evaluation, I have constructed my own text corpus. The corpus is composed of texts written by contemporary Czech authors who provided the texts themselves. The corpus is intentionally small as humans did the evaluation, and evaluating a large set of sentences would be expensive.

In selecting the texts, I tried to capture various characteristics of the texts. Thus, the corpus consists of several different literary genres; the narrators are both male and female; I have included texts written in the present and past tense, and I have chosen texts containing different types of speech, such as direct speech or semi-direct speech. Finally, the corpus, of course, includes both first-person and third-person narratives.

In total, the evaluation data consists of 36 different texts composed of 388 sentences. The numbers divided by narratives are captured in chart \ref{fig:eval-input-numbers}.

\begin{figure}[!ht]
\includegraphics[width=\textwidth]{data/Eval-Input-Numbers.pdf}
\caption{Statistics about the corpus}
\label{fig:eval-input-numbers}
\end{figure}

\section{Annotation Process}

As mentioned in the previous section, human annotators evaluated the sentences manually. The annotators have got: the person the text is converted from, the name of the protagonist, the original text, and the rephrased text. After reading the texts, an annotator could mark one or more statements for each sentence as true.

The possible statements to mark:

\begin{itemize}
	\item The sentence is converted correctly, without grammatical errors, sounds natural, and is unambiguous.
	\item The sentence has lost its unambiguity.
	\item The word order is unnatural, or there are other unnatural sounding elements in the sentence.
	\item The sentence contains grammatical errors.
	\item Some parts of the sentence are converted correctly, and some are not.
	\item The sentence is not converted correctly or not converted at all.
	\item The sentence has not been converted, and that is correct.
\end{itemize}

The evaluation non-binarity also offers a basic error analysis, which gives a basis for discussion about the quality of generated texts.

\section{Results}

Based on the marking, I retrieved the results statistics.

As can be seen in figure \ref{fig:eval-total}, most of the sentences were converted correctly, which also includes not converting the sentence at all (about 66\%). Only 17\% of the 388 sentences were converted incorrectly, partially, or entirely. The remaining 17\% are sentences that were converted correctly. However, the conversion damaged the text's quality (naturalness of expression, unambiguity, or grammar).

\begin{figure}[!ht]
\includegraphics[width=\textwidth]{data/Eval-Total.pdf}
\caption{Results statistics in total}
\label{fig:eval-total}
\end{figure}

The following subsections analyze the results separately for each narrative mode.

\subsection{First-person to Third-person results}

Figure \ref{fig:eval-first-to-third} shows very good results for the first-person narrative to third-person narrative conversion. Less than 5\% of the sentences were converted partially or completely incorrectly. Moreover, about 82\% of the remaining sentences were converted (or not converted) without any quality loss. The most common problem was unnatural word order.

\begin{figure}[!ht]
\includegraphics[width=\textwidth]{data/Eval-First-To-Third.pdf}
\caption{Results statistics for first-person to third-person conversion}
\label{fig:eval-first-to-third}
\end{figure}

In Table \ref{tab:example-first}, I show examples of rephrased sentences with the assigned results.

\begin{table}[!ht]
	\caption{Examples of rephrased sentences from first-person to third-person narrative}
	\label{tab:example-first}
		\begin{tabular}{m{5em}|m{11em}|m{11em}}
			result & original sentence & rephrased sentence \\
			\hline
			Correct & Mám sto chutí sebou trhnout a vysvobodit se z doteku jeho horkých prstů, ale vím, že mi to pouta nedovolí, a i kdybych je neměla, zřejmě by na to moje tělo nenašlo sílu. &  Majka má sto chutí sebou trhnout a vysvobodit se z doteku jeho horkých prstů, ale ví, že jí to pouta nedovolí, a i kdyby je neměla, zřejmě by na to její tělo nenašlo sílu. \\
			Ambiguous & \emph{Odvrátila jsem pohled z~ráje a uviděla Evu.} Následovala mě. & \emph{Odvrátila pohled z ráje a uviděla Evu.} Následovala ji. \\
			Unnatural & Byla jednou z mála lidí, se kterou jsem se hodně sblížil. & Byla jednou z mála lidí, se kterou Gabriel se hodně sblížil. \\
			Grammar error & Sdílela se mnou jídlo. & Sdílela se jí jídlo. \\
			Incorrect & Její obličej nerozeznám, ale slyším, že pláče. & Její obličej nerozezná, ale slyším, že pláče. \\
		\end{tabular}
\end{table}


\subsection{Third-person to First-person results}
On the other hand, figure \ref{fig:eval-third-to-first} illustrates quite a different picture. The third-to-first conversion direction delivers poor performance results. Notice that the number of completely incorrect sentences is high (more than 25\%) even when considering the fact that 68 sentences were not affected by the conversion at all.
The reasons for these poor results are further described in Section \ref{sec:errors}.

\begin{figure}[!ht]
\includegraphics[width=\textwidth]{data/Eval-Third-To-First.pdf}
\caption{Results statistics for third-person to first-person conversion}
\label{fig:eval-third-to-first}
\end{figure}

Examples of the sentences are shown in Table \ref{tab:example-third}.

\begin{table}[!ht]
	\caption{Examples of rephrased sentences from third-person to first-person narrative}
	\label{tab:example-third}
	\begin{flushleft}
		\begin{tabular}{m{5em}|m{11em}|m{11em}}
			result & original sentence & rephrased sentence \\
			\hline
			Correct & Král a královna osaměli. & Král a já jsme osaměli. \\
			Unnatural & Hana se na ni dívala a čekala. & Se na ni jsem dívala a čekala. \\
			Grammar error & Eva byla ráda, že se matka nechala přesvědčit, aby utekly. & Já byla ráda, že se matka nechala přesvědčit, aby jsme utekly. \\
			Incorrect & Hana má pocit, jako by prošvihla nějakou několikaměsíční akci, kde se všichni mezitím seznámili. & Mám pocit, jako by prošvihla nějakou několikaměsíční akci, kde se všichni mezitím seznámili.\\
		\end{tabular}
	\end{flushleft}
\end{table}


\section{Errors and How to Fix Them} \label{sec:errors}

In this section, I provide an error analysis. I examine the causes of the errors and discuss their solvability. The errors are divided into two groups: 1) incorrect conversion and 2) text quality lost.

\subsection{Conversion Errors}

As mentioned in the previous section, most conversion errors come from the third-to-first conversion. Since this direction is much more complicated than the other way, the poorer results are not surprising. In contrast to first-to-third, when converting a text from the third-person narrative, we need to recognize the protagonist who should become the narrator and change all the words whose form depends on this protagonist. This task should be handled by anaphora resolution. Nevertheless, the performance of Aara is poor. Thus, it affects RephrasErIch. The considerable difference between first-to-third and third-to-first results confirms this issue since anaphora resolution is not used in the first-to-third conversion.

A solution to this error is beyond the limits of the tool. Aara would have to be improved or replaced by better performing tool.

\subsection{Text Quality Errors}

\subsubsection{Word order}

Unnatural word order is the most common quality error. It was usually caused by adding or removing words during the conversion. It mostly includes:

\begin{itemize}
	\item \textbf{Adding or removing a subject.} Example: \emph{Adam věří jim to.} The original sentence here was \emph{Věřím jim to.} and the tool added the protagonist's name at the beginning of the sentence. However, the other words should be reordered to sound natural: \emph{Adam jim to věří.}
	\item \textbf{Adding or removing an auxiliary verb.} Example: \emph{Konečně se jsem mohla rozesmát.} The auxiliary verb \emph{jsem} was added to the original sentence \emph{Konečně se mohla rozesmát.}. In Czech, it is natural for reflexive pronoun \emph{se} to come after the auxiliar: \emph{Konečně jsem se mohla rozesmát.}
\end{itemize}

Word order in Czech is not entirely free, but it is variable. It is applied mainly by the role of the sentence parts but also by the rhythmic principle and other circumstances. Often, the word order is determined semantically. Their informativeness and expressive dynamics influence the order of the sentence members, and it also depends on the type of sentence. \cite{vlasin-slovnik} This semantic principle would be challenging to implement in the tool. On the other hand, most words remain in the same order after rephrasing, so I believe that semantics has a relatively minor influence on unnaturalness.

Another principle mentioned is the rhythmic principle. The rhythm of a sentence is mainly influenced by the position of unstressed clitics, such as short forms of personal pronouns (\emph{mě, mi, ho, mu...}), reflexive pronouns (\emph{se, si}) or auxiliary verbs (\emph{jsem, bych, by...}). \cite{vlasin-slovnik}

The relative order of clitics is quite precisely defined. Thus, adding a few rules might help the tool generate more natural sentences.


\subsubsection{Prepositions}

Some prepositions in Czech have both vocalized and non-vocalized forms. The vocalized (longer) versions are used mainly for easier pronunciation. Usually, when the word after the preposition begins with the same consonant or a sequence of several consonants. However, it may also depend on other factors, such as rhythm. \cite{vlasin-slovnik}

For instance, the result of converting the sentence \emph{Táta šel k ní} is \emph{Táta šel k mně}. However, that is unnatural, and it is harder to pronounce -- we would rather write \emph{Táta šel ke mně}. This also applies in vice versa.

Currently, the tool does not change the prepositions when the following word converts. However, this issue would be solvable, at least for cases where the word begins with the same consonant or multiple consonants.


\subsubsection{Proper nouns, unnaturalness, and ambiguity}

When rephrasing, the tool decides in some processes whether to add the proper name of the protagonist instead of the unexpressed subject or pronoun, especially when converting from first to third person. One reason for this addition is to avoid ambiguity. However, this decision is made randomly, leading to the unnaturalness of the text and ambiguity problems.

The \textbf{unnaturalness} occurs when a proper name is inserted in the wrong place in a sentence.

An example may be the following sentence: \emph{Rozhodl jsem se, že půjdu na nákup.} RephrasErIch generated an output: \emph{Rozhodl se, že Adam půjde na nákup.} The sounding would become natural if the proper noun (\emph{Adam}) would stand in the first phrase of the sentence or nowhere. Also, overuse of the name can be unnatural.

In contrast to unnaturalness, \textbf{ambiguity} occurs when a proper name is not inserted. In the first-person narrative, the grammatical person identifies whom the word refers. However, by conversion, we lose this identification. Hence the need to add a proper name to clarify the meaning.

It is a matter of considering where to draw the line between naturalness and ambiguity. Using a proper name in every phrase would lead to perfect unambiguity, but such a sentence would sound highly unnatural. That is the reason why the tool adds the name only with some probability. Nevertheless, it might be helpful to implement a more intelligent decision algorithm, which would consider the position of the phrase in the sentence and further context.



\chapter{Conclusion} \label{chap:zaver}
The aim of this thesis was to explore the topic of person transfer and implement a tool that will transfer texts between the first-person and third-person narratives.
To achieve these goals, I have compiled an overview of the topic from the perspective of literary theory and Czech grammar. Based on this information, I proposed several rules for conversion in both directions. Subsequently, I implemented these rules in the RephrasErIch tool I created. Finally, I evaluated the tool's results on a set of textual data.



\printbibliography

\appendix
\chapter{Attachment} \label{chap:appendix}
The attachment \texttt{rerich.zip} contains:

\begin{itemize}
	\item Directory \texttt{rerich} which consists of:
	\begin{itemize}
		\item Directory \texttt{classes} containing implemented classes mentioned in Chapter \ref{chap:rerich}.
		\item Directory \texttt{enums} with enumeration classes.
		\item Directory \texttt{tools} containg classes providing the communication with external tools.
		\item Files \texttt{erich-rules.py} and \texttt{icher-rules.py} with implementation of the rules described in Chapter \ref{chap:navrh-pravidel}.
		\item File \texttt{utils.py} with some utility functions.
		\item File \texttt{main.py} for running the tool.
	\end{itemize}
	\item File \texttt{aara-rerich.py} which contains a slightly modified implementation of \texttt{Aara}.
\end{itemize}


\end{document}
