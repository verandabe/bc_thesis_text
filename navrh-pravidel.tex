Navrhněte pravidla pro oboustranný převod mezi oběma formami ve spisovné češtině.


\section{First-person to Third-person Rules}
In First-person $\rightarrow$ Third-person conversion direction, I have designed four rules. These four rules cover:
	\begin{itemize}
		\item Personal pronouns replacement
		\item Possesive pronouns replacement
		\item Replacement of contidional verb forms, present verb forms, and conjunctions
		\item Auxiliary verbs replacement or deletion
	\end{itemize}

In this section, I describe each of these rules.

\subsection{Personal pronouns}

This rule covers the conversion of a personal pronoun \emph{já (I)} and its forms. The pronoun can be replaced by:
	\begin{itemize}
		\item another pronoun -- \emph{ona (she)}/\emph{on (he)}
		\item noun -- usually a proper noun, given as the protagonist's name
	\end{itemize}
In both cases, the replacement must be in a corresponding form to keep a sentence grammaticly correct.

\subsubsection{Description of the rule}

\begin{listings}
	Precondition: word.lemma is "já"
	Step 1: decide, if the pronoun would be replaced by proper noun or pronoun
	IF Proper Noun:
		new word = protagonist's name in corresponding case
	ELSE IF the pronoun comes after preposition:
		new\_word = corresponding form of a pronoun "on" starting with "n"
	ELSE:
		new\_world = corresponding form of a pronoun "on" not starting with "n"
\end{listings}

\subsection{Possessive pronouns}

