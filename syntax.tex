\section{Verb and its Person in Czech sentence}

\paragraph{Present indicative tense} is formed by one verb. The ending of a verb depends, among other things, on the person. I show the conjugation on an example in Table \ref{tab:present}.

\begin{table}[!ht]
	\caption{Present tense conjugation in Czech}
	\label{tab:present}
	\begin{center}
		\begin{tabular}{l|r|r}
			person & singular & plural \\
			\hline
			First & já píš\textbf{u} (\emph{I write}) & my píš\textbf{eme} (\emph{we write}) \\
			Second & ty píš\textbf{eš} (\emph{you write}) & vy píš\textbf{ete} (\emph{you write})  \\
			Third & on píš\textbf{e} (\emph{he writes}) & oni píš\textbf{ou} (\emph{they write})  \\
		\end{tabular}
	\end{center}
\end{table}

\paragraph{Future tense} has two possible forms: it can be only one word, in the present tense form of the perfect verb, or construction of a verb \emph{být (to be)} in the future tense and an infinitive of the verb.

In both cases, a verb in the indicative form changes its form depending on the person.

\paragraph{Past tense} exists only in compound form in the Czech language. The compound tense is made up of an auxiliary verb and a participle.

The auxiliary verb is a verb \emph{být (to be)}, and it agrees in person and number with a subject.

The second part of the combination is a past participle, and it carries the meaning. This part agrees in gender and number with a subject.

Nevertheless, the auxiliary verb is not used in the third person.

I show an example of conjugation of verb \emph{psát (write)} in Table \ref{tab:past-tense-conj}.

\begin{table}[!ht]
	\caption{Past tense conjugation in Czech}
	\label{tab:past-tense-conj}
	\begin{center}
		\begin{tabular}{l|r|r}
			person & singular & plural \\
			\hline
			First & já \textbf{jsem} psal (\emph{I wrote}) & my \textbf{jsme} psali (\emph{we wrote}) \\
			Second & ty \textbf{jsi} psal (\emph{you wrote}) & vy \textbf{jste} psali (\emph{you wrote})  \\
			Third & on psal (\emph{he wrote}) & oni psali (\emph{they wrote})  \\
		\end{tabular}
	\end{center}
\end{table}

\paragraph{Passive voice} is composed of the auxiliary verb \emph{být (to be)} and a passive participle of the main verb. Multiple auxiliary verbs can appear in one sentence: one verb to express the passivity and one to express the past.

\paragraph{Conditional Mood} is constructed similarly to the past tense. The difference is the auxiliary verb. The conditional mood is composed of \emph{conditional auxiliar} and a participle. Unlike the past tense, the auxiliar is expressed even in the third person.

An example is shown in Table \ref{tab:cond}.

\begin{table}[!ht]
	\caption{Conditional mood conjugation in Czech}
	\label{tab:cond}
	\begin{center}
		\begin{tabular}{l|r|r}
			person & singular & plural \\
			\hline
			First & já \textbf{bych} psal (\emph{I would write}) & my \textbf{bychom} psali (\emph{we ...}) \\
			Second & ty \textbf{bys} psal (\emph{you would write}) & vy \textbf{byste} psali (\emph{you ...})  \\
			Third & on \textbf{by} psal (\emph{he would write}) & oni \textbf{by} psali (\emph{they ...})  \\
		\end{tabular}
	\end{center}
\end{table}

\section{Special Conjunctions}

By special conjunction, I mean conjunction that changes its form based on the person it refers to. This includes conjunctions \emph{aby} and \emph{kdyby}. The forms are conjugated like the conditional auxiliars.

\section{Personal and Possessive Pronouns}

Besides the verbs and conjunctions, pronouns are also relevant to the narrative. For each person,there are different pronouns whose inflection depends on multiple grammatical categories. Some personal pronouns in Czech may have different forms for the same lemma and tag, depending on their position in the sentence. We distinguish between short, clitic pronouns and long, accented pronouns.

